\documentclass[final,envcountsame]{llncs}


%\usepackage{graphicx} %Pouvoir ins�rer des images
\usepackage{epsfig}
\usepackage{amsmath} %Permet d'utiliser une jolie panoplie de symboles math�matiques
\usepackage{amssymb}
%\usepackage{amsthm}
\usepackage{stmaryrd}
\usepackage[ruled,vlined,linesnumbered]{algorithm2e}
\usepackage{amsfonts}


\def\axisloc#1{\mathcal{LOC}_{#1}}
\def\axislocbis#1#2{\mathcal{LOC}_{#1}^{>#2}}
\def\axislc#1{\mathcal{LC}_{#1}}
\newcommand{\centerness}{Decenter}

\def\OneBall#1{\mathbb{B}^{1}_{#1}}
\def\MaxOneBall#1{\mathbb{M}\OneBall{#1}}

\def\mydist1{D_1}
\def\mydd1{\Omega_1}
\def\mydecent{\mathcal{DC}_1}
\def\dist1{d_1}
\def\Dist1#1{\mydist1(#1)}
\def\DD1#1{\mydd1(#1)}
\def\decent#1{\mydecent(#1)}
\def\cplDist1#1{\mydist1^{cc}(#1)}
\def\cplDD1#1{\mydd1^{cc}(#1)}
\def\cpldecent#1{\mydecent^{cc}(#1)}


\def\mydef#1{{\em #1}}
\def\myem#1{{\em #1}}
\def\myvec#1{\vec{#1}}
\def\quotes#1{``#1''}

\def\birth#1#2{Birth_{#2}(#1)}
\def\death#1#2{Death_{#2}(#1)}
\def\lifespan#1#2{Lifespan_{#2}(#1)}


\def\Nset{\mathbb{N}}
\def\Zset{\Z}
\def\Nset{\mathbb{N}}
\def\Rset{\mathbb{R}}
\def\Z{\mathbb{Z}}
\def\Zhalf{\frac{\mathbb{Z}}{2}}
\def\allfaces#1{\mathbb{F}^{#1}}
\def\subfaces#1#2{\allfaces{#1}_{#2}}
\def\subcomplex#1#2{#1 \preceq #2}
\def\complex#1#2{\subcomplex{#2}{\allfaces{#1}}}
\def\cell#1{\hat{#1}}
\def\cellstrict#1{\hat{#1}^*}
\def\cont#1{\check{#1}}
\def\contstrict#1{\check{#1}^*}
\def\closure#1{#1^-}
\def\princ#1{#1^+}
\def\detach#1#2{#2 \oslash #1}
\def\attach#1#2{\mbox{\it{}Attach\/}(\cell{#1},#2)}
\def\mydim#1{\dim(#1)}
\def\type#1{\mbox{\it{}Dir\/}(#1)}
\def\orient#1{\mbox{\it{}Orient\/}(#1)}
\def\border#1{\mbox{\it{}Border\/}(#1)}
\def\suchthat{\;|\;}
\def\inter{\cap}
\def\union{\cup}
\def\dinterval#1#2{\{#1,\ldots,#2\}}
\newcommand{\Compl}[1]{\overline{#1}}
\def\finproof{\square}
\def\Zhalf{\frac{\mathbb{Z}}{2}}
\newcommand{\Card}[1]{|#1|}

\def\figpath#1{../../Images/EPS/#1}





\begin{document}

\mainmatter

\title{A 3d curvilinear skeletonization algorithm with application to path tracing}
\author{John Chaussard\inst{1} \and Laurent No\"{e}l\inst{2} \and Venceslas Biri\inst{2} \and Michel Couprie\inst{2}}
\institute{Universit\'{e} Paris 13, Sorbonne Paris Cit\'{e}, LAGA, CNRS(UMR 7539), \\ F-93430, Villetaneuse, France \\ \email{chaussard@math.univ-paris13.fr}
\and
Universit\'{e} Paris Est, LABINFO-IGM, A3SI-ESIEE \\ 2, boulevard Blaise Pascal, Cit\'{e} DESCARTES \\ BP 99  93162 Noisy le Grand CEDEX, France \\ \email{laurent.noel@esiee.fr, v.biri@esiee.fr, michel.couprie@esiee.fr}}

\maketitle

\begin{abstract}
What we propose is really great...
\end{abstract}



%%%%%%%%%%%%%%%%%%%%%%%%%%%%%%%%%%%%%%%%%%%%%%%%%%%%%%%%%%
%                                                        %
%                   INTRODUCTION                         %
%                                                        %
%%%%%%%%%%%%%%%%%%%%%%%%%%%%%%%%%%%%%%%%%%%%%%%%%%%%%%%%%%
\section{Introduction}
\label{sec::intro}
Noise in photorealistic computer generated images. Need to accelerate convergence. Add knowledge of the geometry and topology of the scene thanks to discrete geometry. Use the skeleton. Need of a robust curvilinear skeleton that represent topology and approximate geometry. Use the cubical complex framework.

\section{The cubical complex framework}

\subsection{Basic definitions}

In the 3d voxel framework, objects are made of voxels. In the 3d cubical complex framework, objects are made of cubes, squares, lines and vertices. 
Let $\Z$ be the set of integers, we consider the family of sets $\mathbb{F}^1_0$ and $\mathbb{F}^1_1$, such that $\mathbb{F}^1_0 = \{\{a\} \suchthat a \in \Z\}$ and $\mathbb{F}^1_1 = \{ \{a, a+1\} \suchthat a \in \Z\}$. Any subset $f$ of $\Z^n$ such that $f$ is the cartesian product of $m$ elements of $\mathbb{F}^1_1$ and $(n-m)$ elements of $\mathbb{F}^1_0$ is called a face or an \myem{$m$-face} of $\Z^n$, $m$ is the dimension of $f$, we write $dim(f)=m$. A $0$-face is called a \myem{vertex}, a $1$-face is an \myem{edge}, a $2$-face is a \myem{square}, and a $3$-face is a \myem{cube}.

We denote by $\allfaces{n}$ the set composed of all faces in $\Z^n$. Given $m \in \dinterval{0}{n}$, we denote by $\subfaces{n}{m}$ the set composed of all $m$-faces in $\Z^n$.

Let $f \in \allfaces{n}$. We set $\cell{f} = \{g \in \allfaces{n}\vert g \subseteq f\}$, and $\cellstrict{f} = \cell{f} \setminus \{f\}$. Any element of $\cell{f}$ is \myem{a face of $f$}, and any element of $\cellstrict{f}$ is \myem{a proper face of $f$}. We set $\cont{f} = \{g \in \allfaces{n} \vert f \subseteq g\}$, and $\contstrict{f} = \cont{f} \setminus \{f\}$. It is plain that $g \in \cell{f}$ iff $f \in \cont{g}$.

A set $X$ of faces in $\allfaces{n}$ is a \myem{cell}, or \myem{$m$-cell}, if there exists an $m$-face $f \in X$ such that $X=\cell{f}$. The \myem{closure} of a set of faces $X$ is the set $\closure{X}=\union\{\cell{f} \vert f \in X\}$. The set $\Compl{X}$ is $\allfaces{n} \setminus X$. 

\begin{definition}
A finite set $X$ of faces in $\allfaces{n}$ is a \myem{complex} if $X=\closure{X}$, and we write $\complex{n}{X}$. 

Any subset $Y$ of $X$ which is also a complex is a \myem{subcomplex of $X$}, and we write $\subcomplex{Y}{X}$.
\end{definition}


Informally, in 3d, a set of faces $X$ is a complex if each side (square) of a cube of $X$ also belong to $X$, each edge of a square of $X$ also belong to $X$, and each vertex of an edge of $X$ also belong to $X$. 

A face $f \in X$ is \myem{a facet of $X$} if $f$ is not a proper face of any face of $X$. We denote by $\princ{X}$ the set composed of all facets of $X$ . A complex $X$ is \myem{pure} if all its facets have the same dimension. The \myem{dimension of $X$} is $\mydim{X} = \max\{\mydim{f} \suchthat f \in X\}$. If $\mydim{X} = d$, then we say that $X$ is a $d$-complex.



\subsection{From binary images to cubical complex}
\label{sec::binary_to_cubic}

Traditionally, a binary image is a finite subset of $\Zset^n$ (called voxel image when $n=3$). We give now a simple way to transpose such image to the cubical complex framework.

To do so, we associate to each element of $S\subseteq \Zset^n$ an n-face of $\allfaces{n}$. More precisely, let $x=(x_1,...,x_n) \in S$, we define the n-face $\Phi(x) = \{x_1, x_1 + 1\} \times \ldots \times \{x_n, x_n + 1\}$. We can extend the map $\Phi$ to sets: $\Phi(S)=\{\Phi(x) | x \in S\}$. Given a set $S \subset \Zset^n$, we associate to it the cubical complex $\closure{\Phi(S)}$.


\subsection{Thinning: the collapse operation}
The collapse operation is the basic operation for performing homotopic thinning of a complex, and consists of removing two distinct faces $(f,g)$ from a complex $X$ under the condition that they are free:

\begin{definition}
Let $\complex{n}{X}$, and let $f,g$ be two faces of $X$. The face $g$ is \myem{free for $X$}, and the pair $(f,g)$ is \myem{a free pair for $X$} if $(\contstrict{g} \inter X) = \{f\}$. 
\end{definition}

It can be easily seen that if $(f,g)$ is a free pair for a complex $X$, then $f$ is a facet of $X$ and $\dim(g) = \dim(f)-1$.

\begin{definition}
Let $\complex{n}{X}$, and let $(f,g)$ be a free pair for $X$. The complex $X \setminus \{f,g\}$ is an \myem{elementary collapse of $X$}.

Let $\complex{n}{Y}$, the complex $X$ \myem{collapses onto} $Y$ if there exists a sequence of complexes $(X_0,...,X_\ell)$ of $\allfaces{n}$ such that $X=X_0$, $Y=X_\ell$ and for all $i \in \dinterval{1}{\ell}, X_{i}$ is an elementary collapse of $X_{i-1}$. We also say, in this case, that $Y$ \myem{is a collapse of} $X$.
\end{definition}

\section{A parallel directional thinning based on cubical complex}
\label{sec::skeletonization}

\subsection{Removing free pairs in parallel}
In the cubical complex framework, parallel removal of simple pairs can be easily achieved when following simple rules that we give now. First, we need to define the \myem{direction} and the \myem{orientation} of a free face.

Let  $f \in \allfaces{n}$, the \myem{center of $f$} is the center of mass of the points in $f$, that is, $c_f= \frac{1}{\Card{f}}\sum_{a \in f}{a}$. The center of $f$ is an element of $[\Zhalf]^n$, where $\Zhalf$ denotes the set of half integers.
Let $\complex{n}{X}$, let $(f,g)$ be a free pair for $X$, and let $c_f$ and $c_g$ be the respective centers of the faces $f$ and $g$. We denote by $V(f,g)$ the vector $(c_f -c_g)$ of $[\Zhalf]^n$. 

We define a surjective function $\type{}: \allfaces{n} \times \allfaces{n} \rightarrow \dinterval{0}{n-1}$ such that, for all free pairs $(f,g)$ and $(i,j)$ for $X$, $\type{f,g} = \type{i,j}$ if and only if $V(f,g)$ and $V(i,j)$ are collinear. The number $\type{f,g}$ is called the \myem{direction} of the free pair $(f,g)$. Let $(f,g)$ be a free pair, the vector $V(f,g)$ has only one non-null coordinate: the pair $(f,g)$ has a \myem{positive orientation}, and we write $\orient{f,g}=1$, if the non-null coordinate of $V(f,g)$ is positive; otherwise $(f,g)$ has a \myem{negative orientation}, and we write $\orient{f,g}=0$.
%On Fig.~\ref{fig:parallel_collapse}, the free pair $(a,c)$ and the free pair $(d,e)$ have different directions; the free pairs $(a,b)$ and $(d,e)$ have the same direction, but opposite orientations.

Now, we give a property of collapse which brings a necessary and sufficient condition for removing two free pairs of faces in parallel from a complex, while preserving topology.

\begin{proposition}
\label{prop:collpar2}
Let $\complex{n}{X}$, and let $(f,g)$ and $(k,\ell)$ be two distinct free pairs for $X$. The complex $X$ collapses onto $X \setminus \{f,g,k,\ell\}$ if and only if $f \neq k$. 
\end{proposition}

\begin{proof}
If $f=k$, then it is plain that $(k,\ell)$ is not a free pair for $Y=X \setminus \{f,g\}$ as $k=f \notin Y$. Also, $(f,g)$ is not free for $X \setminus \{k,\ell\}$. If $f\neq k$, then we have $g \neq \ell$, $\contstrict{g} \inter X = \{f\}$ ($g$ is free for $X$) and $\contstrict{\ell} \inter X = \{k\}$ ($\ell$ is free for $X$). Thus, we have $\contstrict{\ell} \inter Y = \{k\}$ as $\ell\neq g$ and $k\neq f$ and $(k,\ell)$ is a free pair for $Y$. $\finproof$
\end{proof}

From Prop.~\ref{prop:collpar2}, the following corollary is immediate.

\begin{corollary}
\label{cor:parallel_removal}
Let $\complex{n}{X}$, and let $(f_1,g_1) \ldots (f_m,g_m)$ be $m$ distinct free pairs for $X$ such that, for all $a,b \in
\dinterval{1}{m}$ (with $a \neq b$), $f_a \neq f_b$. The complex $X$ collapses onto $X \setminus \{f_1,g_1 \ldots f_m,g_m\}$.
\end{corollary}

Considering two distinct free pairs $(f,g)$ and $(i,j)$ for $\complex{n}{X}$ such that $\type{f,g} = \type{i,j}$ and $\orient{f,g} = \orient{i,j}$, we have $f \neq i$.  From this observation and Cor.~\ref{cor:parallel_removal}, we deduce the following property.

\begin{corollary}
\label{cor:parallel_removal_v2}
Let $\complex{n}{X}$, and let $(f_1,g_1) \ldots (f_m,g_m)$ be $m$ distinct free pairs for $X$ having all the same direction and the same orientation. The complex $X$ collapses onto $X \setminus \{f_1,g_1 \ldots f_m,g_m\}$. 
\end{corollary}



\subsection{A directional parallel thinning algorithm}

Intuitively, we want the algorithm to remove free faces of a complex \quotes{layer by layer} and avoid having unequal thinning of the input complex. Therefore, we want each execution of the algorithm to remove free faces located on the border of the input complex. 
We define $\border{X}$ as the set all faces belonging to a free pair for $X$. We now introduce Alg.~\ref{algo::pardircollapse_v0}, a directional parallel thinning algorithm.

On a single execution of the main loop of Alg.~\ref{algo::pardircollapse_v0}, only faces located on the border of the complex are removed (l.~\ref{pc:07}). Thanks to corollary \ref{cor:parallel_removal_v2}, we remove faces with same direction and orientation in parallel (l.~\ref{pc:08}).

\begin{algorithm}[h!]
\label{algo::pardircollapse_v0}
\SetKwFor{ForAll}{for all}{do}{end}
\caption[ParDirCollapse]{$ParDirCollapse(X,W,\ell)$}
\KwData{A cubical complex $\complex{n}{X}$, a subcomplex
  $\subcomplex{W}{X}$ which represents faces of $X$ which should not
  be removed, and $\ell \in \Nset$, the number of layers of free faces
  which should be removed from $X$} 
\KwResult{A cubical complex}
\lnl{pc:01}\While{there exists free faces in $X \setminus W$ and $\ell > 0$}{
\lnl{pc:02}  $L=\border{X}^-$\;
\lnl{pc:03}  \For{$t=0 \rightarrow n-1$}{
\lnl{pc:04}    \For{$s=0 \rightarrow 1$}{
\lnl{pc:05}      \For{$d=n \rightarrow 1$}{
\lnl{pc:06}        $E=\{(f,g)$ free for $X \suchthat g \notin W,$\\
                      $\;$~~~~~~~$\type{f,g}=t,$ $\orient{f,g}=s,$ $\mydim{f}=d\}$\;
\lnl{pc:07}        $G=\{(f,g) \in E \suchthat f \in L$ and $g \in L\}$\;
\lnl{pc:08}        $X=X \setminus G$\;
                 }
               }
             }
\lnl{pc:09}  $l = l-1$\;
	   }
\lnl{pc:10} \Return $X$\;
\end{algorithm}



Different definitions of the orientation and direction can be given, corresponding to different order of removal of free faces in the complex and leading to different results.
Algorithm~\ref{algo::pardircollapse_v0} can be implemented to run in linear time complexity (proportionally to the number of faces of the complex). Indeed, checking if a face is free or not may be easily done in constant time. Moreover, when a free pair $(f,g)$ is removed from the input complex, it is sufficient to scan the elements of $(\cellstrict{f} \union \contstrict{g})$ in order to find new free faces.



\section{Aspect preservation during thinning: a parameter-free method}
\label{sec::aspect_no_param}
Algorithm \ref{algo::pardircollapse_v0} does not necessarily preserves "geometrical information" from the original object in the resulting skeleton. For example, if the input was a filled human shape (one connected component, no tunnel nor cavity) and that $W=\emptyset$ and $l=+\inf$, then the result of Alg.~\ref{algo::pardircollapse_v0} would be a simple vertex. However, when one wants to preserve "geometrical information" from the original shape in the skeleton, then one expects to obtain a "stickman" from a human shape. It is however important to not retain "too much information" during thinning in order to avoid noisy branches in the skeleton.

In the following, we will see a new method in the cubical complex, requiring no user input, for obtaining a curvilinear skeleton yielding satisfactory visual properties. This will be achieved by adding, in the set $W$ of Alg.~\ref{algo::pardircollapse_v0}, some faces from the original object during the thinning process.

\subsection{The lifespan of a face}
In the following, we define more notions in the cubical complex. The first one we present is the \myem{death date} of a face.

\begin{definition}
Let $f \in \complex{n}{X}$, the \myem{death date of $f$ in $X$}, denoted by $\death{f}{X}$, is the smallest integer $d$ such that $f \notin ParDirCollapse(X, \emptyset, d)$. 
\end{definition}

The death date of a face indicates how many layers of free faces should be removed from a complex $X$, using Alg.~\ref{algo::pardircollapse_v0}, before removing completely the face from $X$. We now define the \myem{birth date} of a face:

\begin{definition}
Let $f \in \complex{n}{X}$, the \myem{birth date of $f$ in $X$}, denoted by $\birth{f}{X}$, is the smallest integer $b$ such that either $f$ is a facet of $ParDirCollapse(X, \emptyset, b)$, or either $f \notin ParDirCollapse(X, \emptyset, b)$.
\end{definition}

The birth date indicates how many layers of free faces must be removed from $X$ with Alg.\ref{algo::pardircollapse_v0} before transforming $f$ into a facet of $X$ (we consider a face "lives" when it is a facet).
Finally, we can define the \myem{lifespan} of a face :

\begin{definition}
Let $f \in \complex{n}{X}$, the lifespan of $f$ in $X$ is the integer 

\begin{math}
\lifespan{f}{X} = \left\{
\begin{array}{ll}
+\infty & \text{if $\death{f}{X}=+\infty$}\\
\death{f}{X} - \birth{f}{X} & \text{otherwise}
\end{array}
\right.
\end{math}
\end{definition}

These three values are dependant on the order of direction and orientation chosen for Alg.~\ref{algo::pardircollapse_v0}.

The lifespan of a face $f$ of $X$ indicates how many "rounds" this face "survives" as a facet in $X$, when removing free faces with Alg.~\ref{algo::pardircollapse_v0}. 
The lifespan is a good indicator of how important a face can be in an object. Typically, higher the lifespan is, and more representative of an object's visual feature the face is. The lifespan, sometimes called \myem{saliency}, was used in \cite{Liu2009} (with the name "medial persistence") in order to propose a thinning algorithm in cubical complexes based on two parameters.



\subsection{Distance map, opening function and decenterness map}
\label{subsec::distmap}
In addition to the lifespan of a face, the proposed homotopic thinning method uses information on distance between faces in order to decide if a face should be kept safe from deletion. We define hereafter the various notions needed for this, based on distance in the voxel framework.

Two points $x,y \in \Zset^n$ are \myem{$1$-neighbours} if the Euclidean distance between $x$ and $y$ is equal or inferior to $1$ (also called direct neighbours). A \myem{$1$-path from $x$ to $y$} is a sequence $\mathcal{C}=(z_0,...,z_k)$ of points of $\Zset^n$ such that $z_0=x$, $z_k=y$, and for all $j \in [1;k]$, $z_j$ and $z_{j-1}$ are $1$-neighbours. The length of $\mathcal{C}$ is $k$. Remark that $d_1$ is indeed the $6$-distance in the 3d voxel framework.

We set $\dist1(x,y)$ as the length of the shortest $1$-path from $x$ to $y$. Let $S \subset \Zset^n$, we set, for all $x \in Zset^n$, the map $\Dist1{S}: \Zset^n \rightarrow \Nset$ such that $\Dist1{S}(x) = \displaystyle \min_{y \in \Compl{S}} \dist1(x,y)$. 

The \myem{maximal $1$-ball of $S$ centered on $x$} is the set $\MaxOneBall{S}(x)=\{y \in \Zset^n | \dist1(x,y) < \Dist1{S}(x)\}$.
We set, for all $x \in S$, the map $\DD1{S}: \Zset^n \rightarrow \Nset$ such that $\DD1{S}(x) = \displaystyle \max_{x \in \MaxOneBall{S}(y)} \Dist1{S}(y)$: this value indicates the radius of the largest maximal $1$-ball contained in $S$ and containing $x$. If $x \in \Compl{S}$, we set $\DD1{S}(x) = 0$. The map $\DD1{S}$ is known as the opening function of $S$ (based on the $1$-distance): it allows to compute efficiently results of morphological openings by balls of various radius, and gives information on the local thickness of an object.

Given $S \subset \Zset^n$, the value of $\DD1{S}(x)$ of every $x \in S$ can be naively computed by performing successive morphological dilations of values of the map $\Dist1{S}$. 
%TODO Parler du fait que l'on a une version dans la th�se, et une version plus am�lior� dans th�se et public avenir

Finally, we define the \myem{decenterness map} :

\begin{definition}
\label{def::decenterness_map}
Given $S \subset \Zset^n$, the \myem{decenterness map of $S$} is the map $\decent{S} = \DD1{S} - \Dist1{S}$.
\end{definition}

In order to extend all these previous maps defined in $\Zset^n$ to the cubical complex framework, let us introduce the map $\Phi^{-1}$, inverse of the bijective map $\Phi : \subfaces{n}{n} \rightarrow \Zset^{n}$ defined in Sec.~\ref{sec::binary_to_cubic}. It is used to project any $n$-face of $\allfaces{n}$ into $\Zset^n$. We indifferently use $\Phi^{-1}$ as a map from $\subfaces{n}{n}$ to $\Zset^n$, and as a map from $\mathcal{P}(\subfaces{n}{n})$ to $\mathcal{P}(\Zset^n)$.

Given $Y \subset \allfaces{n}$, we define the map $\cplDist1{Y}: \allfaces{n} \rightarrow \Nset$ as follows: for all $f \in \allfaces{n}$,

\bigskip
\begin{math}
\cplDist1{Y}(f) = \left\{
\begin{array}{ll}
\Dist1{\Phi^{-1}(Y \inter \subfaces{n}{n})}(Phi^{-1}(f)) & \text{if $f$ is an $n$-face}\\
\displaystyle \max_{g \in \contstrict{f} \inter \subfaces{n}{n}} \cplDist1{Y}(g) & \text{else}
\end{array}
\right.
\end{math}

Informally, if $f$ is a 3-face, then $\cplDist1{Y}(f)$ is the length of the shortest 1-path between the voxel \quotes{corresponding} to $f$ and the set of voxels corresponding to $Y$.
The same way, we define $\cplDD1{Y}$ and $\cpldecent{Y}$.


\subsection{Parameter-free thinning based on the lifespan, opening function and decenterness}
As previously said, we add edges to the set $W$ of Alg.~\ref{algo::pardircollapse_v0} in order to retain, in the resulting curvilinear skeleton, important edges from the original object. Given a cubical complex $X$, if an edge of $X$ has a high decenterness value for $X$, then it is probably located too close to the border of $X$ and does not represent an interesting visual feature to preserve.
%TODO ref vers image de decenterness
On the other hand, if an edge has a high lifespan for $X$, then it means it was not removed quickly, after becoming a facet, by the thinning algorithm and might represent some precious visual information on the original object. An idea would be to keep, during thinning, all edges whose lifespan is superior to the decenterness value. Unfortunately, this strategy produces skeletons with many spurious branches in surfacic areas of the original object.

We can identify surfacic areas of a complex as zones where squares have a high lifespan. Therefore, in order to avoid spurious branches in surfacic areas, we need to make it harder for edges to appear in these zones. It can be achieved by deciding that an edge will be kept safe from deletion by the thinning algorithm if its lifespan is superior to the decenterness value plus the lifespan of squares \quotes{around} this edge. This leads us to proposing Alg.~\ref{algo::3d_curvskel}. 

\begin{algorithm}[ht]
\label{algo::3d_curvskel}
\caption[CurvilinearSkeleton]{$CurvilinearSkeleton(X)$}
\KwData{A cubical complex $\complex{3}{X}$}
\KwResult{A cubical complex $\complex{3}{Y}$}
\lnl{curvskel:01} $W=\{f \in X | \lifespan{f}{X} > \cpldecent{X}(f) + \birth{f}{X} - \cplDist1{X}(f)$ and $\mydim{f}=1\}$\;
\Return $ParDirCollapse(X,W,+\infty)$\;
\end{algorithm}

In order to understand what was realised on line \ref{curvskel:01} of Alg.~\ref{algo::3d_curvskel}, we might point out that the birth date of an edge corresponds to the highest death date of the squares containing this edge. Moreover, the map $\cplDist1{X}$ gives, for all 3-faces of $X$, their death date (as the thinning algorithm naturally follows this map to eliminate cubes from a 3-complex). Therefore, for an edge $f$ of $X$, $\cplDist1{X}(f)$ informs us on the highest death date of cubes containing $f$, also equal to the highest birth date of squares containing $f$. In conclusion, $\birth{f}{X} - \cplDist1{X}(f)$ is an approximation of the lifespan of the squares containing $f$.




\section{Application to path tracing}

Path tracing is a global illumination algorithm that is able to render photo-realistic images of a virtual scene viewed by a camera. The idea behind the algorithm is very elegant: for each pixel of the image, throw rays through the pixel that will bounce on the scene to accumulate lighting exchanges. In order to get fast convergence the rays must be efficiently distributed through the scene, toward the lights. Our idea is to use a curvilinear skeleton of the void of the scene to guide the rays in the correct direction. We start by defining some useful concepts related to path tracing and we present our skeleton based path tracing algorithm.

%\subsection{Basic definitions of radiometry}
%
%Let $S$ be a set of surfaces of $\Rset^3$ that don't overlap and $S_L \subset S$ a set of surface lights. $S$ is called the scene.
%
%The basic radiometric quantity is the \myem{radiance} $L(x \rightarrow \Theta)$ that expresses the quantity of light energy produced by the surface point $x \in \Rset^3$ toward direction $\Theta \in \Omega_x$ per unit of time per unit area per unit solid angle (expressed in $W.m^{-2}.sr^{-1}$).
%This is the quantity that the eye actually "sees" and that must be computed for each pixel of the final image.
%
%% image radiance
%
%We can define another quantity related to radiance called \myem{incoming radiance} $L(x \leftarrow \Phi)$ that represents the radiance that reach $x$ from direction $\Phi$.
%
%In the void, incoming radiance can be written in terms of radiance as follow:
%
%\begin{equation*}
%L(x \leftarrow \Phi) = L(r(x, \Phi) \rightarrow -\Phi)
%\end{equation*}
%
%$r(x, \Phi)$ is the surface point seen by $x$ in the direction $\Phi$. It can be defined formally by:
%
%\begin{align*}
%r(x, \Phi) &= x + t_{min}\Phi \\
%t_{min} &= min \lbrace t \in (0, +\infty]~ |~ x + t\Phi \in S \rbrace
%\end{align*}
%
%If $t_{min} = \infty$ then $r(x, \Phi)$ is undefined and $L(x \leftarrow \Phi) = 0$. We call $r$ the \myem{raytrace} function. In practice it is implemented by throwing a ray in the scene and looking for the first intersection.
%
%A third quantity related to radiance is \myem{emitted radiance} $L_e(x \rightarrow \Theta)$. It is not null if $x$ is a point of a light source ($x \in S_L$). In practice the emitted radiance is given with the set $S_L$ as an input of the algorithm.
%
%The radiance is the solution of an equation called the Rendering Equation:
%
%\begin{align*}
%L(x \rightarrow \Theta) &= L_e(x \rightarrow \Theta) + \int_{\Phi \in \Omega_x} f_r(x, \Theta \leftrightarrow \Phi) L(x \leftarrow \Phi) cos(N_x, \Phi) d\omega_\Phi \\
%&= L_e(x \rightarrow \Theta) + L_r(x \rightarrow \Theta)
%\end{align*}
%
%This equation expresses an intuitive idea: the radiance produced by the point $x$ in the direction $\Theta$ is the sum of the emitted radiance and the reflected radiance.  The reflected radiance is computed by integrating the incoming radiance scaled by a factor $f_r(x, \Theta \leftrightarrow \Phi) cos(N_x, \Phi)$ over all the possible directions of the hemisphere around $x$. $N_x$ is the surface normal at point $x$.
%
%The $f_r$ function is called the \myem{bidirectional reflectance function} (BRDF) and express the exchange of radiance between the incoming direction $\Phi$ and the output direction $\Theta$. Intuitively it represents the material properties of the surface at point $x$. A mirror doesn't reflect light the same manners a wall do, the BRDF encodes that.
%
%It's actually the rendering equation that a path tracer try to resolve. Let $O$ be the origin of the camera and $I$ the image we want to compute (a rectangle embedded in $\Rset^3$). Then the path tracing algorithm compute an approximation of the radiances $L(O \leftarrow \vec{OP})$, $\forall P \in I$.

\subsection{The path tracing}

Let $O$ be the origin of the camera in the scene. For each pixel $P$ of the final image we search for the nearest intersection point $x$ of the ray $(O, \overrightarrow{OP} = -\Theta)$ with the scene. At this point, to obtain the luminosity, we must solve the rendering equation \cite{Ka86}. The rendering equation is a recursive integral equation. The integrand of the equation contains the radiance function that we must compute:

\begin{align*}
L(x \rightarrow \Theta) = L_e(x \rightarrow \Theta) + \int_{\Phi \in \Omega_x} f_r(x, \Theta \leftrightarrow \Phi) L(r(x, \Phi) \rightarrow -\Phi) cos(N_x, \Phi) d\omega_\Phi
\end{align*}

$L$ is the radiance (a radiometric quantity that represents luminosity) from point $x$ toward direction $\Theta$. $L_e$ is the emitted radiance that is not zero only on light sources. $r(x, \Phi)$ is the nearest visible point from $x$ in the direction $\Phi$. The $f_r$ function is called the \myem{bidirectional reflectance function} (BRDF) and express the exchange of radiance between the incoming direction $\Phi$ and the output direction $\Theta$. Intuitively it represents the material properties of the surface at point $x$. A mirror doesn't reflect light the same manners a wall do, the BRDF encodes that.

This equation expresses an intuitive idea: the radiance $L$ produced by the point $x$ in the direction $\Theta$ is the sum of the emitted radiance and the reflected radiance.  The reflected radiance is computed by integrating the incoming radiance scaled by a factor $f_r(x, \Theta \leftrightarrow \Phi) cos(N_x, \Phi)$ over all the possible directions of the hemisphere around $x$. $N_x$ is the surface normal at point $x$.

In practice it is impossible to compute anatically the integral, it has to be estimated.
The most commonly used method is the Monte-Carlo integration that provides the following estimator for the integral:
\begin{equation*}
\langle L_r(x \rightarrow \Theta) \rangle = \frac{f_r(x, \Theta \leftrightarrow \Phi) L(r(x, \Phi) \rightarrow -\Phi) cos(N_x, \Phi)}{p(\Phi)}
\end{equation*}

$p$ is a probability density function (pdf) that is used to sample $\Phi$, a direction of the hemisphere $\Omega_x$. This estimator is unbiased, it means that the expected value $E[\langle L_r(x \rightarrow \Theta) \rangle]$ is equal to $L_r(x \rightarrow \Theta)$. The variance of the estimator express its quality.
The variance of the estimator depends on the choosen pdf $p$. Actually the best is to choose a pdf that matches the shape of the function to integrate (ie. give high density to samples that have high values for the function and low density to samples that have low values).

This strategy of choosing an adapted pdf is called \myem{importance sampling} and is over-used in global illumination to improve the convergence of the algorithms.

The implantation of $L$ is as follow: first we sample a random direction $\Phi$ based on the pdf $p$. Then we apply the estimator that call $L$ recursively to compute $L_r$. Then we return the sum of the emitted radiance $L_e$ and the result for $L_r$. The recursion stops when a maximal number of bounces have been reach.

This algorithm is called for each pixel of the final image to compute. To improve the quality of the final image we can throw multiple rays accross each pixel and compute the mean of the results.

In the radiance algorithm the $\Phi$ direction is sampled on $\Omega_x$ based on the pdf $p$. A bad pdf picks directions that don't reach the light before the end of the recursion and results in a lot of noise in the final image. We will discuss the choice of $p$ in the next section.

\subsection{Skeleton based importance sampling}

The key idea behind our method is that light travels in the void of the scene. A curvilinear skeleton of that void gives us enough informations to compute some importance directions that tells us where the light comes from. Given that directions we can construct a efficient pdf $p_{skel}$ and guide our rays by sampling the hemispheres with $p_{skel}$

Here is the integrand of the integral estimated with Monte-Carlo:

\begin{equation*}
f_r(x, \Phi, \Theta) L(x \leftarrow \Phi) cos(N_x, \Phi)
\end{equation*}

It is a product of three functions that can be efficiently sample individually but that are hard to sample when they are combined. The most common strategy used to sample $\Omega_x$ is to use the BRDF (combined with the cosine) because we know it: it is given as an input of the algorithm, for exemple with textures.

$L$ is rarelly used because we cannot evaluate it in constant time. Remember that it represents the distribution of light in the scene. It takes high values when $\Phi$ carries a lot of energy. Out method gives a way to sample according to $L$.

\subsubsection{Construction of the importance points}

The skeleton of the void is computed with our algorithm and is converted to a graph (the nodes are 3D positions and the edges represents the topology of the scene), given as an input of our path tracing. We then compute a set of points called \myem{importance points}. These points will be used to sample $\Omega_x$ in the path tracing algorithm. For each node of the skeleton $n$, one importance point $imp_n$ is computed.

Let $L$ be a light of the scene ($L \in S_L$) and $n_L$ the shortest visible node by $L$. We first compute a tree of shortest paths along the skeleton with $n_L$ as root. We use the Dijkstra algorithm with a distance based on visibility from $n_L$: We put $d(n, m) = 1$ if $m$ is visible from $n_L$ and $d(n, m) = 10$ if not. It results that the paths that are enlighted by $L$ will have a shorter distance and will guide us to the light faster than the dark ones.

Let $n$ be a node of the skeleton and $V_n$ the set of visible nodes from $n$ along the shortest path toward $n_L$. The importance point $imp_n$ associated to $n$ is computed by taking the barycenter of $V_n$.

\subsubsection{Sampling according to $L$}

Given a point $x$ and a direction $\Theta$ we want to compute $L(x \rightarrow \Theta)$ and then sample the hemisphere $\Omega_x$. We search for the nearest skeleton node $n$ and its importance point $imp_n$. We sample the hemisphere with a power-cosine pdf center on $\overrightarrow{ximp_n}$:

\begin{equation*}
p_{skel}(\Phi) = \frac{s + 1}{2\pi} cos^s \alpha
\end{equation*}

with $\alpha$ the angle between $\overrightarrow{ximp_n}$ and $\Phi$ and $s$ a parameter called skeleton strength. The more $s$ is high, the more we sample close $\overrightarrow{ximp_n}$.

\clearpage
\subsection{Results and discussion}

We present the results of our algorithm applied to four scenes. the first row shows our result and the second row shows a standard path tracing with the same input parameters. On each picture a MSE (mean square error) value with a reference image is shown.

\begin{center}
\hspace*{-40px}
\includegraphics[scale=0.2]{images/four_scenes.png}
\end{center}

We observe a net decreasing of the noise for our algorithm, especially in dark areas. The MSE values are lower for the top pictures, meaning that we converge faster to the reference image than standard path tracing does.

We presented an application of our skeletonization algorithm for global illumination. Our strategy works well, it reduces noise by taking into account the distribution of energy in the scene. One of the problem is that we take into account only one importance direction for sampling, we totally ignore the BRDF and the cosine factor. A solution called Multiple Importance Sampling allow us to combine different strategies into one. We will not describe this strategy because it's not directly related to skeletonization, but it has allowed us to improve our results and remove some artifacts of our method.




\bibliographystyle{splncs03}
\bibliography{./bibli}


\end{document}





